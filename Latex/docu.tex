\documentclass[a4paper]{article}
\usepackage[a4paper,left=3cm,right=2cm,top=2.5cm,bottom=2.5cm]{geometry}
\usepackage[utf8]{inputenc}
\usepackage{amsmath}
\usepackage{algorithm}
\floatname{algorithm}{Algorisme} % Cambia "Algorithm" por "Algorisme"

\usepackage{algpseudocode}
\usepackage{hyperref}
\usepackage{graphicx}
\usepackage{float}
\usepackage{array}

\title{\textbf{Intel·ligència Artificial:\\
		Pràctica SBC}}
\author{\emph{Guillem Cabré, Carla Cordero, Hannah Röber}}
\date{Curs 2024-25, Quadrimestre de tardor}

\renewcommand*\contentsname{Continguts}
\renewcommand{\figurename}{Figura}
\renewcommand{\tablename}{Taula}

\begin{document}
	
	\begin{titlepage}
		\clearpage\maketitle
		\thispagestyle{empty}
	\end{titlepage}
	
	\tableofcontents
	\clearpage
	
	\section{Identificació}
	
	\subsection{Descripció del problema}
	
	Els museus d’art són una de les principals atraccions turístiques a moltes ciutats, oferint l’oportunitat de gaudir d’obres mestres de diverses èpoques i estils. En el món real, els museus poden ser gegantescos, com per exemple el Museu del Louvre a Paris, el Museu Britànic a Londres, o el Museu del Prado a Madrid. Un dels problemes principals en visitar un museu és la limitació de temps per veure tot el que s’ofereix. Això genera la necessitat de desenvolupar aplicacions que ajudin els visitants a organitzar la seva visita de manera personalitzada. \\
	
	Els museus solen estar dividits en sales organitzades per artistes o per temes. Les obres d’art poden ser seleccionades en funció de diverses característiques, com l’any de creació, l’època, l’estil, l’autor, la sala on es troben, la temàtica, les dimensions, la complexitat o la importància de l’obra. A més, s’han de considerar els tipus de visitants i les seves preferències, com ara el temps disponible, el tipus de grup (individual, família, grup petit o gran) i el nivell de coneixement d’art. \\
	
	L’objectiu és crear una eina capaç d’ajustar la visita segons les preferències i característiques dels visitants, tenint en compte restriccions com el temps diari disponible i l’organització lògica de les sales.
	
	\subsection{Viabilitat del SBC}
	
	Abans de res, cal analitzar si el problema pot ser gestionat de manera òptima per un Sistema Basat en el Coneixement (SBC). Per plantejar-ho, ens fem la següent pregunta: seria possible que una persona experta en art dissenyés rutes personalitzades per a cada perfil de visitant d’un museu? La resposta és clara: sí. De fet, aquest ha estat el mètode tradicional, on les visites guiades han estat dissenyades per persones expertes en art. El nostre objectiu és modularitzar i automatitzar aquest procés, emulant el raonament d’un expert per dissenyar rutes òptimes entre les obres d’art d’un museu. \\
	
	A partir del problema plantejat, és evident que podem abstraure coneixement utilitzant una ontologia. Aquesta ontologia ens permetrà representar i relacionar els diferents conceptes clau del domini, com ara les característiques de les obres d’art, les preferències dels visitants i l’organització espacial del museu. A més, mitjançant l’ús de regles basades en aquest coneixement, podrem traçar un camí que ens guiï des del plantejament del problema fins a la seva solució de manera efectiva. \\
	
	Podem concloure que aquest enfocament ens permetrà construir un sistema capaç d’assistir els visitants de manera personalitzada i eficient.
	
	\subsection{Fonts del Coneixement}
	
	Tot el coneixement que integrarà el programa estarà organitzat en tres blocs principals:  
	\begin{itemize}  
		\item \textbf{Ontologia}: Aquest bloc ens permetrà relacionar els diferents conceptes de manera estructurada i organitzada. A més, serà fonamental per definir els atributs i les relacions entre aquests conceptes, oferint una base sòlida per al raonament del sistema.  
		
		\item \textbf{Instàncies}: Mitjançant CLIPS, es representarà tot aquell coneixement estàtic que es mantindrà invariable per a tots els visitants. Aquestes instàncies inclouran elements com les sales del museu, les obres d'art, els artistes, els estils pictòrics i fins i tot rutes predefinides del museu. Les rutes predefinides es descriuen amb més detall en una secció posterior.  
		
		\item \textbf{Visitant}: Per personalitzar l’experiència, el sistema recollirà informació sobre el visitant mitjançant un breu qüestionari. Aquest permetrà identificar les seves característiques i preferències, facilitant així la generació d’una ruta personalitzada que s’ajusti al màxim a les seves necessitats i interessos.  
	\end{itemize}
	
	\subsection{Objectius}
	
	Els objectius d’aquest projecte són:
	\begin{enumerate}
		\item Analitzar i modelar el domini dels museus per construir una ontologia adequada.
		\item Desenvolupar un sistema basat en coneixement capaç de proposar rutes de visita personalitzades.
		\item Implementar una solució en CLIPS que permeti simular visites segons les preferències i el temps disponible.
		\item Avaluar la solució mitjançant casos de prova representatius que validin la qualitat de les rutes generades.
	\end{enumerate}
	Aquests objectius s’assoleixen seguint una metodologia incremental basada en prototipatge, amb èmfasi en la modularització i la justificació de les decisions preses.
	
	\subsection{Resultats Esperats}
	
	En finalitzar aquest projecte, s’espera obtenir els següents resultats:
	
	\begin{itemize}
		\item \textbf{Sistema Basat en Coneixement (SBC) operatiu}: Un sistema capaç de generar rutes personalitzades per als visitants del museu, tenint en compte les seves preferències i el temps disponible.
		
		\item \textbf{Ontologia del domini dels museus}: Una representació formal dels conceptes i les relacions del domini, que permeti al sistema raonar de manera eficient.
		
		\item \textbf{Generació automàtica de rutes personalitzades}: Un procés automatitzat que utilitzi les preferències del visitant per crear una ruta òptima que maximitzi l'experiència de la visita.
		
		\item \textbf{Avaluació mitjançant casos de prova}: La validació del sistema es durà a terme mitjançant proves representatives que permetin avaluar l'eficàcia i l'eficiència de les rutes generades.
	\end{itemize}
	
	Tot i que aquest projecte ha estat dissenyat com a part d'una tasca universitària, l'objectiu principal és aprendre a implementar Sistemes Basats en Coneixement (SBC). Encara que el projecte no es lliurarà a cap museu perquè en faci ús, ens agradaria que servís com a fonament per a futurs projectes viables en la vida real, amb l'objectiu de personalitzar rutes de visita en museus.
	
	\newpage
	\section{Conceptualització}
	
	La conceptualització d'aquest projecte es basa en la identificació, definició i descripció dels elements fonamentals que conformen el domini. Aquesta etapa és clau, ja que permet establir les bases conceptuals necessàries per al desenvolupament del sistema, garantint així una implementació coherent amb els objectius inicials. \\
	
	La interrelació entre aquests elements és fonamental, ja que el sistema ha de ser capaç de generar recomanacions personalitzades de rutes de visita tenint en compte les preferències dels visitants, les característiques de les obres d'art i la distribució de les sales. Per assolir aquest objectiu, s'han definit atributs específics per cada element que permeten capturar informació rellevant, com la complexitat d'una obra, la durada disponible de la visita o la preferència per certs estils artístics. Aquesta informació es recull de forma automàtica a través d'un conjunt de preguntes inicials, que pretenen conèixer el perfil del visitant i els seus interessos. \\
	
	Mitjançant la definició clara i precisa dels conceptes del domini, es redueixen les possibles ambigüitats del disseny i es faciliten els mecanismes de personalització de les rutes. \\
		
	\subsection{Elements del domini}
	\label{sec:elements_del_domini}
	
	Els elements del domini fonamenten el SBC. Cada element té una funció específica i es defineix mitjançant atributs que en descriuen les seves propietats rellevants. Aquesta classificació permet organitzar la informació de forma estructurada i facilitar les relacions entre ells. \\
	
	\noindent \textbf{Característiques d'una visita}
	\begin{itemize}
		\item \texttt{Definició}: Representen les persones que fan una visita al museu.
		\item \texttt{Atributs}:
		\begin{itemize}
			\item \texttt{Nombre persones}: Número de persones que formen el grup de visita. Pot ser individual (una persona), o un grup.
			\item \texttt{Nombre museus visitats}: Nombre de museos visitats l'últim any, d'aquí derivará el càlcul de coneixement de la visita sobre museus.
			\item \texttt{Durada diària}: Temps disponible per dia (hores).
			\item \texttt{Dies totals}: Nombre de dies disponibles per a la visita.
			\item \texttt{Familia}: Hi ha nens al grup.
			\item \texttt{Preferència d'artista}: Artista d'interès per la visita.
			\item \texttt{Preferència d'estil}: Estil d'interès per la visita.
		\end{itemize}
	\end{itemize}
		
		
	\noindent \textbf{Característiques d'un autor}
	\begin{itemize}
		\item \texttt{Definició}: Artistes creadors de les obres del museu.
		\item \texttt{Atributs}:
		\begin{itemize}
			\item \texttt{Nom}: Identificació de l’artista.
			\item \texttt{Nacionalitat}
			\item \texttt{Estil}: Interval històric de temps en què va crear obres.
			
		\end{itemize}
	\end{itemize}
	
	
	\noindent \textbf{Característiques d'una obra d'art}
	\begin{itemize}
		\item \texttt{Definició}: Objecte del museu amb unes certes característiques.
		\item \texttt{Atributs}:
			\begin{itemize}
				\item \texttt{Títol}: Nom de l’obra (p. ex., "La Gioconda").
				\item \texttt{Autor}: Referència a l’autor que ha creat l'obra.
				\item \texttt{Any de creació}
				\item \texttt{Estil}: Període històric en què es va crear (p. ex., Renaixement).
				\item \texttt{Sala}: Ubicació al museu.
				\item \texttt{Dimensions}
				\item \texttt{Complexitat}: Calculada a partir de les mides de l'obra.
			\end{itemize}
	\end{itemize}


	\noindent \textbf{Característiques d'una sala d'art}
	\begin{itemize}
		\item \texttt{Definició}: Espai on s'exposen un conjunt d'obres d'art.
		\item \texttt{Atributs}:
		\begin{itemize}
			\item \texttt{Museu al que pertany}
		\end{itemize}
		\item \texttt{Subclasses}:
		\begin{itemize}
			\item \texttt{Sala Artista}: Sala dedicada a exposar les obres d'un artista concret.
			\item \texttt{Sala Estil}: Sala dedicada a exposar les obres d'un estil concret.
		\end{itemize}
	\end{itemize}
	
	\subsubsection{Selecció d'obres}
	
	Tot i que aquest projecte no representa un museu real, tampoc la seva representació al sistema ho serà. Aquesta decisió es va prendre pel fet que la complexitat de seleccionar un gran nombre d'obres d'un museu, com el Prado o el Louvre, era massa elevada, especialment considerant que no estem davant d'un problema de la vida real. Si haguéssim considerat necessari utilitzar un conjunt més ampli d'obres, hauríem pogut implementar un \textit{scraper} en \textit{Python} per obtenir un fitxer amb totes les instàncies necessàries i inserir-les a l'ontologia. No obstant això, vam decidir optar per una solució més senzilla que s'ajustés millor als objectius del projecte.\\
	
	El nostre museu estarà compost per un nombre limitat d'estils artístics. Concretament, hem seleccionat tres estils principals: Barroc, Contemporani i Renaixentista. Tant les obres com els autors que s'exposaran al museu pertanyen a aquests estils. Com que cap membre del grup és expert en art, vam optar per fer ús d'una eina externa per facilitar la selecció. Concretament, vam utilitzar \textit{ChatGPT} per obtenir una llista de 5 artistes rellevants de cada període i, per a cadascun d'ells, una selecció de 5 de les seves obres més representatives.\\
	
	Un cop rebuda la llista suggerida, la vam revisar per identificar possibles errors i adaptar-la a les nostres necessitats. Com a resultat, vam obtenir una configuració final composta per $3$ estils, $15$ autors ($3 \times 5$) i $75$ obres ($3 \times 5 \times 5$). Aquesta selecció ens permet treballar amb un conjunt de dades manejable però suficientment representatiu per als nostres objectius de prova i validació del sistema.
	
	\subsection{Divisió en subproblemes}
	\label{sec:Subproblemes}
	
	\subsubsection{Problema concret: Recollir dades de la visita}
	
	Abans de començar a buscar una solució pel problema, és necessari recopilar la informació rellevant sobre el perfil dels visitants. Aquesta informació es recull mitjançant un conjunt de preguntes que permeten identificar:
	\begin{itemize}
		\item Les característiques generals del grup: tipus (individual, família, grup petit o grup gran) i coneixement d’art (baix, mitjà o alt).
		\item Les limitacions temporals de la visita, incloent-hi el nombre de dies disponibles i la durada màxima per jornada.
		\item Les preferències personals dels visitants, com ara estils artístics i autors favorits.
	\end{itemize}
	
	Aquestes dades són essencials per personalitzar l’experiència del visitant i garantir que el pla de visita s’ajusti als seus interessos i necessitats.
	
	\paragraph{Preguntes inicials per al visitant:}
	A continuació, es presenta el conjunt de preguntes que es fan als visitants per tal de recollir aquesta informació:
	\begin{itemize}
		\item \textbf{Sobre el grup:}
		\begin{itemize}
			\item Quantes persones sou? 
			\begin{itemize}
				\item Resposta numèrica
			\end{itemize}
			\item Hi ha nens al grup? 
			\begin{itemize}
				\item Sí
				\item No
			\end{itemize}
			\item Quants museus heu visitat l'últim any? 
			\begin{itemize}
				\item Resposta numèrica
			\end{itemize}
		\end{itemize}
		
		\item \textbf{Sobre el temps de visita:}
		\begin{itemize}
			\item Quants dies teniu disponibles per visitar el museu?
			\item Quantes hores podeu dedicar-hi cada dia?
		\end{itemize}
		
		\item \textbf{Sobre les preferències:}
		\begin{itemize}
			\item Hi ha algun estil/època artística que us interessi especialment?
			\item Hi ha algun autor que us agradaria veure?
		\end{itemize}
		Per aquestes preguntes es mostra el conjunt d'opcions disponible al museu, també una opció per si no es vol marcar preferència.
	\end{itemize}

	
	\subsubsection{Abstracció: analitzar els interessos de la visita}
	
	Un cop recollides les dades, es crea el problema abstracte. El sistema analitza les preferències i característiques proporcionades pels visitants. Aquesta anàlisi inclou:
	\begin{itemize}
		\item Determinar si la visita serà curta o llarga en funció de les hores i dies que l'usuari ha indicat.
		\item Guardar si la visita serà individual, o d'un grup gran o petit, en funció del nombre de persones que la formen.
		\item Determinar el grau de coneixement en art de la visita.
		\item Emmagatzemar les preferències d'estil de la visita.
	\end{itemize}
	
	El resultat d’aquest procés és el problema abstracte i servirà com a base per planificar la visita.
	
	
	\subsubsection{Associació: Generar una solució abstracte}

	Partint del problema abstracte, el sistema s’encarrega de trobar un conjunt d'obres d'art tenint en compte les necessitats trobades. Això inclou:
	\begin{itemize}
		\item Seleccionar un conjunt d'obres seguint els interessos dels visitants.
		\item Determinar un conjunt d'obres tenint en compte el temps de la visita i estimant el temps necessari per observar cada obra.
		\item Assignar un ordre lògic a les obres seleccionades, minimitzant els salts innecessaris entre sales.
	\end{itemize}
	
	L’objectiu és proporcionar un recorregut coherent i fàcil de seguir que maximitzi l’aprofitament del temps dels visitants.
	
	\subsubsection{Refinament: Creació d'una solució concreta}
	
	Adaptem el recorregut, assegurant que la durada total de cada jornada no superi les hores disponibles. Finalment, el sistema genera una proposta completa i personalitzada per al visitant, amb una llista d'obres agrupades per sales. \\
	
	El procés d'adaptació del recorregut seleccionat a les preferències del visitant serà el següent:
	\begin{itemize}
		\item Afegir totes les obres de l'autor preferit pel visitant, si aquest té una preferència específica.
		\item Afegir totes les obres de l'estil pictòric preferit del visitant, si en té.
		\item Ajustar el temps de la següent manera: 
		\begin{itemize}
			\item Si el temps estimat de la ruta és superior al temps que el visitant vol destinar, eliminarem obres segons el seu grau de complexitat i el coneixement del visitant. Si el visitant té un coneixement elevat, es treuran les obres menys complexes; i a l'inrevés, si el coneixement és baix, es retiraran les obres més complexes.
			\item Si el temps estimat de la ruta és inferior al temps disponible, afegirem obres seguint la mateixa regla que anteriorment. Si el visitant és expert en art, afegirem obres amb una complexitat elevada. I si no ho és, afegirem obres de complexitat reduïda.
		\end{itemize}
	\end{itemize}
	
	Un cop tenim la selecció final, es retornaran les obres ordenades per sales. D'aquesta manera, l'usuari podrà optimitzar el seu temps sense canviar de sala innecessàriament. \\
	
	Cal tenir en compte que, com a temps mitjà per obra, hem establert 15 minuts. En futures versions del sistema, seria interessant ajustar el temps dedicat a cada obra depenent de la seva complexitat i, a més, en funció de qui la visiti. No es triga el mateix a observar la Mona Lisa per part d'un nen petit que per part d'un expert en art, ja que aquest últim es fixarà en cada pinzellada.
	
	\subsection{Identificació de conceptes principals}
	
	\subsubsection{Tipus de visites}
	Les visites al museu es poden classificar segons diversos criteris relacionats amb el perfil dels visitants i les seves necessitats:
	\begin{itemize}
		\item \texttt{Tipus de grup}:
		\begin{itemize}
			\item Individual
			\item Grup petit (2-5 persones)
			\item Grup gran (més de 5 persones)
		\end{itemize}
		\item \texttt{Durada de la visita}:
		\begin{itemize}
			\item Curta: 5 hores o menys per tota la visita
			\item Llarga: més de 5 hores
		\end{itemize}
		\item \texttt{Coneixement artístic}:
		\begin{itemize}
			\item Baix: ha visitat pocs museus, per tant, suposem que coneix poques obres o autors.
			\item Mitjà: coneix algunes èpoques o estils.
			\item Alt: coneix moltes obres i autors, pot visitar obres més complexes
		\end{itemize}
	\end{itemize}
	
	\subsubsection{Disseny del museu}
	El museu tracta principalment sobre tres èpoques clau de la història de l'art: {Renaixement, Barroc i Art Modern}. De cadascuna d'aquestes èpoques comptem amb un gran ventall d'artistes que han estat representatius. És per això que aquest està organitzat en diferents sales que agrupen totes les obres artístiques segons criteris específics.
	
	Tenim sales dedicades a cadascuna d'aquestes èpoques. A més, també hi ha sales dedicades a alguns dels artistes més importants d'aquests grans moments històrics. Algunes de les sales disponibles són:
	
	\begin{itemize}
		\item \texttt{Sales Temàtiques: } 
		\begin{itemize}
			\item Renaixement
			\item Barroc
			\item Art Modern
		\end{itemize}
	\end{itemize}
	
	\begin{itemize}
		\item \texttt{Sales sobre artistes: } 
		\begin{itemize}
			\item Claude Monet: Artista d'Art Modern
			\item Wassily Kandinsky: Artista d'Art Modern
			\item ArtemisiaGentileschi: Artista del Barroc
			\item Rembrandt Van Rijn: Artista del Barroc
		\end{itemize}
	\end{itemize}
	
	\newpage
	\section{Formalització}
	
	\subsection{Desenvolupament de la ontologia}
	
	\subsubsection*{Artista}
	
	\subsubsection*{Museu}
	
	\subsubsection*{Obra d'Art}
	
	\subsubsection*{Ruta}
	
	\subsubsection*{Sala}
	
	\subsubsection*{Visita}
	
	\subsection{Justificació de la metodologia de resolució}
	
	La metodologia utilitzada per resoldre el problema es basa en una divisió clara del sistema en diferents mòduls independents. Aquesta elecció es justifica per les següents raons:
	
	\begin{itemize}
		\item \texttt{Estructura modular:} El problema de generar rutes personalitzades per als visitants d’un museu es compon de diverses parts:
		\begin{itemize}
			\item Recopilació de dades dels visitants.
			\item Abstracció i classificació de la visita.
			\item Associació d’obres i sales en funció de les preferències i limitacions.
			\item Refinament i optimització final de les rutes.
		\end{itemize}
		Aquesta divisió permet abordar cada part de forma independent, reduint la complexitat del desenvolupament i assegurant que cada component funcioni correctament abans d’integrar-lo al sistema complet.
		
		\item \texttt{Flexibilitat i manteniment:} L'ús de mòduls clarament definits facilita la modificació i ampliació del sistema en el futur. Per exemple, es poden afegir nous estils d’art, obres o criteris de personalització sense afectar altres parts del sistema.
		
		\item \texttt{Evolució incremental:} El sistema construeix la solució de manera progressiva, començant per la recopilació de dades, seguint amb l’anàlisi i generació d’una ruta inicial, i acabant amb el refinament i l’optimització. Aquest enfocament escalonat garanteix que cada pas afegeixi un nivell de detall fins a obtenir una solució ajustada a les necessitats del visitant.
		
	\end{itemize}
	
	En conjunt, aquesta metodologia assegura que el sistema sigui escalable i fàcil de gestionar, de la mateixa forma que, en un futur, permet aplicar nous canvis en els criteris seleccionats.

	
	\newpage
	\section{Implementació}
	
	La implementació del sistema s'ha estructurat en diverses fases, cadascuna centrada en una part específica del desenvolupament. A continuació, es detallen les principals etapes i les tasques realitzades en cadascuna:
	
	\subsection{Creació de l’ontologia i instàncies}
	
	L’ontologia ha estat dissenyada per representar de forma estructurada els conceptes clau del domini: les classes i els atributs per modelar el museu. Un cop definida aquesta amb Protegé, incloent els elements del domini explicats en apartats anteriors, s'han creat les instàncies estàtiques del museu, buscant un conjunt representatiu de: sales obres d'art, artistes i èpoques.
	
	\subsection{MAIN}
	\begin{itemize}
		\item Mòdul principal que actua com a punt d’entrada.
		\item Activa la seqüència de preguntes per al visitant i inicialitza el flux d’execució del sistema.
	\end{itemize}
	
	
	\subsection{Mòdul "recopilacion": recopilació de dades}
	
	Aquest mòdul inclou regles i funcions encarregades d'obtenir informació inicial dels visitants a través de preguntes interactives. Les dades recollides inclouen:
	\begin{itemize}
		\item Característiques del grup: Nombre de persones, presència de nens i experiència prèvia (nombre de museus visitats).
		\item Duració de la visita: Dies disponibles i hores per dia.
		\item Preferències: Estils artístics i artistes favorits.
	\end{itemize}
	Les respostes es processen i s'emmagatzemen com a fets per a l'ús posterior del sistema.
	
	
	\subsection{Mòdul "abstraccion": d’abstracció}
	
	Un cop recollides les dades, aquest mòdul transforma la informació inicial en un problema abstracte. Les tasques realitzades inclouen:
	\begin{itemize}
		\item Classificació del grup: Individual, grup petit o grup gran.
		\item Determinació del coneixement artístic: Baix, mitjà o alt, basat en el nombre de museus visitats.
		\item Càlcul de la durada de la visita: Curta o llarga, segons els dies i hores disponibles.
	\end{itemize}
	Aquestes classificacions serveixen com a base per a la generació de rutes.
	
	\subsection{Mòdul de matching}
	
	Aquest mòdul genera una primera ruta pel visitant basada en les seves preferències i característiques, és a dir, en el seu interès en art i en el seu coneixement. \\
	
	Aquest mòdul prioritza les obres i sales més rellevants per al visitant.
	
	\subsection{Mòdul de refinament}
	
	Aquest mòdul ajusta la ruta inicial per adaptar-la millor a les necessitats del visitant:
	\begin{itemize}
		\item Afegir obres d'artistes i estils preferits no inclosos inicialment.
		\item Afegir o treure obres en funció del temps disponible i el seu coneixement.
	\end{itemize}
	El resultat és una ruta final personalitzada i agrupada de forma lògica.
	
	\subsection{Resultats generats}
	
	Un cop completats tots els passos, el sistema presenta al visitant una ruta final amb les obres seleccionades, ordenades per sales i sessions diàries. \\
	
	Aquest resultat intenta ajustar-se al perfil del visitant i maximitzar l’aprofitament del temps.
	
	
	\newpage
	\section{Jocs de Prova}
	
\end{document}