\documentclass[a4paper]{article}
\usepackage[a4paper,left=3cm,right=2cm,top=2.5cm,bottom=2.5cm]{geometry}
\usepackage[utf8]{inputenc}
\usepackage{amsmath}
\usepackage{algorithm}
\floatname{algorithm}{Algorisme} % Cambia "Algorithm" por "Algorisme"

\usepackage{algpseudocode}
\usepackage{hyperref}
\usepackage{graphicx}
\usepackage{float}
\usepackage{array}

\title{\textbf{Intel·ligència Artificial:\\
		Pràctica SBC}}
\author{\emph{Guillem Cabré, Carla Cordero, Hannah Röber}}
\date{Curs 2024-25, Quadrimestre de tardor}

\renewcommand*\contentsname{Continguts}
\renewcommand{\figurename}{Figura}
\renewcommand{\tablename}{Taula}

\begin{document}
	
	\begin{titlepage}
		\clearpage\maketitle
		\thispagestyle{empty}
	\end{titlepage}
	
	\tableofcontents
	\clearpage
	
	\section{Introducció}
	
	\subsection{Descripció del problema}
	
	Els museus d’art són una de les principals atraccions turístiques a moltes ciutats, oferint l’oportunitat de gaudir d’obres mestres de diverses èpoques i estils. En el món real, els museus poden ser gegantescos, com per exemple el Museu del Louvre a Paris, el Museu Britànic a Londres, o el Museu del Prado a Madrid. Un dels problemes principals en visitar un museu és la limitació de temps per veure tot el que s’ofereix. Això genera la necessitat de desenvolupar aplicacions que ajudin els visitants a organitzar la seva visita de manera personalitzada. \\
	
	Els museus solen estar dividits en sales organitzades per artistes o per temes. Les obres d’art poden ser seleccionades en funció de diverses característiques, com l’any de creació, l’època, l’estil, l’autor, la sala on es troben, la temàtica, les dimensions, la complexitat o la importància de l’obra. A més, s’han de considerar els tipus de visitants i les seves preferències, com ara el temps disponible, el tipus de grup (individual, família, grup petit o gran) i el nivell de coneixement d’art. \\
	
	L’objectiu és crear una eina capaç d’ajustar la visita segons les preferències i característiques dels visitants, tenint en compte restriccions com el temps diari disponible i l’organització lògica de les sales.
	
	\subsection{Objectius}
	
	Els objectius d’aquest projecte són:
	\begin{enumerate}
		\item Analitzar i modelar el domini dels museus per construir una ontologia adequada.
		\item Desenvolupar un sistema basat en coneixement capaç de proposar rutes de visita personalitzades.
		\item Implementar una solució en CLIPS que permeti simular visites segons les preferències i el temps disponible.
		\item Avaluar la solució mitjançant casos de prova representatius que validin la qualitat de les rutes generades.
	\end{enumerate}
	Aquests objectius s’assoleixen seguint una metodologia incremental basada en prototipatge, amb èmfasi en la modularització i la justificació de les decisions preses.
	
	\subsection{Viabilitat del SBC}
	
	Abans de res, cal analitzar si el problema pot ser gestionat de manera òptima per un Sistema Basat en el Coneixement (SBC). Per plantejar-ho, ens fem la següent pregunta: seria possible que una persona experta en art dissenyés rutes personalitzades per a cada perfil de visitant d’un museu? La resposta és clara: sí. De fet, aquest ha estat el mètode tradicional, on les visites guiades han estat dissenyades per persones expertes en art. El nostre objectiu és modularitzar i automatitzar aquest procés, emulant el raonament d’un expert per dissenyar rutes òptimes entre les obres d’art d’un museu. \\
	
	A partir del problema plantejat, és evident que podem abstraure coneixement utilitzant una ontologia. Aquesta ontologia ens permetrà representar i relacionar els diferents conceptes clau del domini, com ara les característiques de les obres d’art, les preferències dels visitants i l’organització espacial del museu. A més, mitjançant l’ús de regles basades en aquest coneixement, podrem traçar un camí que ens guiï des del plantejament del problema fins a la seva solució de manera efectiva. \\
	
	Podem concloure que aquest enfocament ens permetrà construir un sistema capaç d’assistir els visitants de manera personalitzada i eficient.
	
	\subsection{Fonts del Coneixement}
	
	Tot el coneixement que integrarà el programa estarà organitzat en tres blocs principals:  
	\begin{itemize}  
		\item \textbf{Ontologia}: Aquest bloc ens permetrà relacionar els diferents conceptes de manera estructurada i organitzada. A més, serà fonamental per definir els atributs i les relacions entre aquests conceptes, oferint una base sòlida per al raonament del sistema.  
		
		\item \textbf{Instàncies}: Mitjançant CLIPS, es representarà tot aquell coneixement estàtic que es mantindrà invariable per a tots els visitants. Aquestes instàncies inclouran elements com les sales del museu, les obres d'art, els artistes, els estils pictòrics i fins i tot rutes predefinides del museu. Les rutes predefinides es descriuen amb més detall en una secció posterior.  
		
		\item \textbf{Visitant}: Per personalitzar l’experiència, el sistema recollirà informació sobre el visitant mitjançant un breu qüestionari. Aquest permetrà identificar les seves característiques i preferències, facilitant així la generació d’una ruta personalitzada que s’ajusti al màxim a les seves necessitats i interessos.  
	\end{itemize}  
	
	
	\section{Conceptualització}
	
	En aquest apartat, identifiquem els conceptes principals necessaris per modelar el domini i implementar el sistema:
	
	\subsection{Elements del domini (?)}
	\label{sec:elements_del_domini}
	
	Característiques d'una visita
	
	
	Característiques d'un visitant
	\begin{itemize}
		\item \texttt{Definició}: Representen les persones que fan una visita al museu.
		\item \texttt{Atributs}:
		\begin{itemize}
			\item \texttt{Tipus}: Individual, família, grup petit, grup gran.
			\item \texttt{Coneixement}: Qualitatiu (baix, mitjà, alt).
			\item \texttt{Preferències}: Llista d’interessos (autors, èpoques, estils, temàtiques).
			\item \texttt{Durada diària}: Temps disponible per dia (hores).
			\item \texttt{Dies totals}: Nombre de dies disponibles per a la visita.
		\end{itemize}
	\end{itemize}
	
	Característiques d'una obra d'art
	\begin{itemize}
		\item \texttt{Definició}: Objecte del museu amb unes certes característiques.
		\item \texttt{Atributs}:
			\begin{itemize}
				\item \texttt{Títol}: Nom de l’obra (p. ex., "La Gioconda").
				\item \texttt{Autor}: Referència a l’autor (p. ex., Leonardo da Vinci).
				\item \texttt{Any de creació}
				\item \texttt{Època}: Període històric en què es va crear (p. ex., Renaixement).
				\item \texttt{Estil/escola/corrent}: Moviment artístic o corrent (p. ex., Barroc).
				\item \texttt{Sala}: Ubicació al museu (p. ex., "Sala 1").
				\item \texttt{Temàtica}: Categoria general (p. ex., retrat, paisatge).
				\item \texttt{Dimensions}
				\item \texttt{Complexitat}: Calculada a partir de les mides de l'obra.
				\item \texttt{Rellevància}: Valor qualitatiu (alta, mitjana, baixa).
			\end{itemize}
	\end{itemize}
	
	Característiques d'un autor
	\begin{itemize}
		\item \texttt{Definició}: Artistes creadors de les obres del museu.
		\item \texttt{Atributs}:
		\begin{itemize}
			\item \texttt{Nom}: Identificació de l’artista.
			\item \texttt{Nacionalitat}
			\item \texttt{Època}: Interval de temps en què va crear obres.
			\item \texttt{Estil/escola/període}: Moviments artístics amb què es relaciona.
		\end{itemize}
	\end{itemize}
	
	\subsection{Divisió en subproblemes}
	
	\subsubsection{Recollir dades de la visita}
	
	Abans de començar a buscar una solució pel problema, és necessari recopilar la informació rellevant sobre el perfil dels visitants. Aquesta informació es recull mitjançant un conjunt de preguntes que permeten identificar:
	\begin{itemize}
		\item Les característiques generals del grup: tipus (individual, família, grup petit o grup gran) i coneixement d’art (baix, mitjà o alt).
		\item Les preferències personals dels visitants, com ara èpoques, estils artístics, autors favorits o temàtiques d’interès.
		\item Les limitacions temporals de la visita, incloent-hi el nombre de dies disponibles i la durada màxima per jornada.
	\end{itemize}
	
	Aquestes dades són essencials per personalitzar l’experiència del visitant i garantir que el pla de visita s’ajusti als seus interessos i necessitats.
	
	\paragraph{Preguntes inicials per al visitant:}
	A continuació, es presenta el conjunt de preguntes que es fan als visitants per tal de recollir aquesta informació:
	\begin{itemize}
		\item \textbf{Sobre el grup:}
		\begin{itemize}
			\item Quin tipus de grup sou? 
			\begin{itemize}
				\item Individual
				\item Família
				\item Grup petit (2-5 persones)
				\item Grup gran (més de 5 persones)
			\end{itemize}
			\item Quin coneixement d’art teniu? 
			\begin{itemize}
				\item Baix
				\item Mitjà
				\item Alt
			\end{itemize}
		\end{itemize}
		
		\item \textbf{Sobre el temps de visita:}
		\begin{itemize}
			\item Quants dies teniu disponibles per visitar el museu?
			\item Quantes hores podeu dedicar-hi cada dia?
		\end{itemize}
		
		\item \textbf{Sobre les preferències:}
		\begin{itemize}
			\item Hi ha alguna època artística que us interessi especialment?
			\item Quin estil o corrent artístic preferiu?
			\item Hi ha algun autor que us agradaria veure?
			\item Us interessa alguna temàtica en particular?
		\end{itemize}
		Per aquestes preguntes es mostra el conjunt d'opcions disponible al museu, també una opció per si no es vol marcar preferència
	\end{itemize}

	
	\subsubsection{Analitzar els interessos de la visita}
	
	Un cop recollides les dades, el sistema analitza les preferències i característiques proporcionades pels visitants. Aquesta anàlisi inclou:
	\begin{itemize}
		\item Comparar les preferències amb les característiques de les obres disponibles al museu.
		\item Filtrar les obres que compleixin amb els criteris especificats pel visitant, donant prioritat a aquelles amb major rellevància o que s’ajustin millor a les seves preferències.
	\end{itemize}
	
	El resultat d’aquest procés és un conjunt inicial d’obres seleccionades, que servirà com a base per planificar la visita.
	
	
	\subsubsection{Organitzar la visita}

	Després de seleccionar les obres adequades, el sistema s’encarrega de trobar un recorregut. Això inclou:
	\begin{itemize}
		\item Seleccionar un conjunt d'obres seguint els interessos dels visitants.
		\item Estimar el temps necessari per observar cada obra, tenint en compte la seva rellevància, complexitat i les preferències del visitant.
		\item Dividir les obres en sessions diàries, assegurant que la durada total de cada jornada no superi les hores disponibles.
		\item Assignar un ordre lògic a les obres seleccionades, minimitzant els salts innecessaris entre sales.
	\end{itemize}
	
	L’objectiu és proporcionar un recorregut coherent i fàcil de seguir que maximitzi l’aprofitament del temps dels visitants.
	
	\subsubsection{Presentar la visita}
	
	Finalment, el sistema genera una proposta completa i personalitzada pel visitant, amb una llista d'obres seleccionades amb el temps assignat a cadascuna i l’ordre de visita suggerit, agrupat per sales i sessions diàries.

	
	\subsection{Identificació de conceptes principals}
	
	\subsubsection{Tipus de visites}
	Les visites al museu es poden classificar segons diversos criteris relacionats amb el perfil dels visitants i les seves necessitats:
	\begin{itemize}
		\item \texttt{Tipus de grup}:
		\begin{itemize}
			\item Individual
			\item Família
			\item Grup petit (2-5 persones)
			\item Grup gran (més de 5 persones)
		\end{itemize}
		\item \texttt{Durada de la visita}:
		\begin{itemize}
			\item 1 dia
			\item 2 dies
			\item 3 o més dies
		\end{itemize}
		\item \texttt{Coneixement artístic}:
		\begin{itemize}
			\item Baix: coneix poques obres o autors.
			\item Mitjà: coneix algunes èpoques o estils.
			\item Alt: coneix moltes obres i autors, amb preferències clares.
		\end{itemize}
	\end{itemize}
	
	\subsubsection{Sales del museu}
	El museu està organitzat en diferents sales que agrupen les obres segons criteris específics. Algunes de les sales disponibles són:
	\begin{itemize}
		\item \texttt{Sala 1: } 
		\begin{itemize}
			\item Obres destacades:
			\item Autors:
		\end{itemize}
	\end{itemize}
	
	\subsubsection{Quadres}
	
\end{document}