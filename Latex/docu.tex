\documentclass[a4paper]{article}
\usepackage[a4paper,left=3cm,right=2cm,top=2.5cm,bottom=2.5cm]{geometry}
\usepackage[utf8]{inputenc}
\usepackage{amsmath}
\usepackage{algorithm}
\floatname{algorithm}{Algorisme} % Cambia "Algorithm" por "Algorisme"

\usepackage{algpseudocode}
\usepackage{hyperref}
\usepackage{graphicx}
\usepackage{float}
\usepackage{array}

\title{\textbf{Intel·ligència Artificial:\\
		Pràctica SBC}}
\author{\emph{Guillem Cabré, Carla Cordero, Hannah Röber}}
\date{Curs 2024-25, Quadrimestre de tardor}

\renewcommand*\contentsname{Continguts}
\renewcommand{\figurename}{Figura}
\renewcommand{\tablename}{Taula}

\begin{document}
	
	\begin{titlepage}
		\clearpage\maketitle
		\thispagestyle{empty}
	\end{titlepage}
	
	\tableofcontents
	\clearpage
	
	\section{Introducció}
	
	\subsection{Descripció del problema}
	
	\subsection{Característiques del domini}
	
	Obres d'art, visitants...\\
	
	\subsection{Objectius}
	
	
	
	\section{Conceptualització}
	
	En aquest apartat, identifiquem els conceptes principals necessaris per modelar el domini i implementar el sistema:
	
	\subsection{Elements del domini (?)}
	
	Característiques d'una visita
	
	
	Característiques d'un visitant
	\begin{itemize}
		\item \texttt{Definició}: Representen les persones que fan una visita al museu.
		\item \texttt{Atributs}:
		\begin{itemize}
			\item \texttt{Tipus}: Individual, família, grup petit, grup gran.
			\item \texttt{Coneixement}: Qualitatiu (baix, mitjà, alt).
			\item \texttt{Preferències}: Llista d’interessos (autors, èpoques, estils, temàtiques).
			\item \texttt{Durada diària}: Temps disponible per dia (hores).
			\item \texttt{Dies totals}: Nombre de dies disponibles per a la visita.
		\end{itemize}
	\end{itemize}
	
	Característiques d'una obra d'art
	\begin{itemize}
		\item \texttt{Definició}: Objecte del museu amb unes certes característiques.
		\item \texttt{Atributs}:
			\begin{itemize}
				\item \texttt{Títol}: Nom de l’obra (p. ex., "La Gioconda").
				\item \texttt{Autor}: Referència a l’autor (p. ex., Leonardo da Vinci).
				\item \texttt{Any de creació}
				\item \texttt{Època}: Període històric en què es va crear (p. ex., Renaixement).
				\item \texttt{Estil/escola/corrent}: Moviment artístic o corrent (p. ex., Barroc).
				\item \texttt{Sala}: Ubicació al museu (p. ex., "Sala 1").
				\item \texttt{Temàtica}: Categoria general (p. ex., retrat, paisatge).
				\item \texttt{Dimensions}
				\item \texttt{Complexitat}: Calculada a partir de les mides de l'obra.
				\item \texttt{Rellevància}: Valor qualitatiu (alta, mitjana, baixa).
			\end{itemize}
	\end{itemize}
	
	Característiques d'un autor
	\begin{itemize}
		\item \texttt{Definició}: Artistes creadors de les obres del museu.
		\item \texttt{Atributs}:
		\begin{itemize}
			\item \texttt{Nom}: Identificació de l’artista.
			\item \texttt{Nacionalitat}
			\item \texttt{Època}: Interval de temps en què va crear obres.
			\item \texttt{Estil/escola/període}: Moviments artístics amb què es relaciona.
		\end{itemize}
	\end{itemize}
	
	\subsection{Divisió en subproblemes}
	
	\subsubsection{Recollir dades de la visita}
	
	Abans de començar a buscar una solució pel problema, és necessari recopilar la informació rellevant sobre el perfil dels visitants. Aquesta informació es recull mitjançant un conjunt de preguntes que permeten identificar:
	\begin{itemize}
		\item Les característiques generals del grup: tipus (individual, família, grup petit o grup gran) i coneixement d’art (baix, mitjà o alt).
		\item Les preferències personals dels visitants, com ara èpoques, estils artístics, autors favorits o temàtiques d’interès.
		\item Les limitacions temporals de la visita, incloent-hi el nombre de dies disponibles i la durada màxima per jornada.
	\end{itemize}
	
	Aquestes dades són essencials per personalitzar l’experiència del visitant i garantir que el pla de visita s’ajusti als seus interessos i necessitats.
	
	\paragraph{Preguntes inicials per al visitant:}
	A continuació, es presenta el conjunt de preguntes que es fan als visitants per tal de recollir aquesta informació:
	\begin{itemize}
		\item \textbf{Sobre el grup:}
		\begin{itemize}
			\item Quin tipus de grup sou? 
			\begin{itemize}
				\item Individual
				\item Família
				\item Grup petit (2-5 persones)
				\item Grup gran (més de 5 persones)
			\end{itemize}
			\item Quin coneixement d’art teniu? 
			\begin{itemize}
				\item Baix
				\item Mitjà
				\item Alt
			\end{itemize}
		\end{itemize}
		
		\item \textbf{Sobre el temps de visita:}
		\begin{itemize}
			\item Quants dies teniu disponibles per visitar el museu?
			\item Quantes hores podeu dedicar-hi cada dia?
		\end{itemize}
		
		\item \textbf{Sobre les preferències:}
		\begin{itemize}
			\item Hi ha alguna època artística que us interessi especialment?
			\item Quin estil o corrent artístic preferiu?
			\item Hi ha algun autor que us agradaria veure?
			\item Us interessa alguna temàtica en particular?
		\end{itemize}
		Per aquestes preguntes es mostra el conjunt d'opcions disponible al museu, també una opció per si no es vol marcar preferència
	\end{itemize}

	
	\subsubsection{Analitzar els interessos de la visita}
	
	Un cop recollides les dades, el sistema analitza les preferències i característiques proporcionades pels visitants. Aquesta anàlisi inclou:
	\begin{itemize}
		\item Comparar les preferències amb les característiques de les obres disponibles al museu.
		\item Filtrar les obres que compleixin amb els criteris especificats pel visitant, donant prioritat a aquelles amb major rellevància o que s’ajustin millor a les seves preferències.
	\end{itemize}
	
	El resultat d’aquest procés és un conjunt inicial d’obres seleccionades, que servirà com a base per planificar la visita.
	
	
	\subsubsection{Organitzar la visita}

	Després de seleccionar les obres adequades, el sistema s’encarrega de trobar un recorregut. Això inclou:
	\begin{itemize}
		\item Seleccionar un conjunt d'obres seguint els interessos dels visitants.
		\item Estimar el temps necessari per observar cada obra, tenint en compte la seva rellevància, complexitat i les preferències del visitant.
		\item Dividir les obres en sessions diàries, assegurant que la durada total de cada jornada no superi les hores disponibles.
		\item Assignar un ordre lògic a les obres seleccionades, minimitzant els salts innecessaris entre sales.
	\end{itemize}
	
	L’objectiu és proporcionar un recorregut coherent i fàcil de seguir que maximitzi l’aprofitament del temps dels visitants.
	
	\subsubsection{Presentar la visita}
	
	Finalment, el sistema genera una proposta completa i personalitzada pel visitant, amb una llista d'obres seleccionades amb el temps assignat a cadascuna i l’ordre de visita suggerit, agrupat per sales i sessions diàries.

	
	\subsection{Identificació de conceptes principals}
	
	\subsubsection{Tipus de visites}
	Les visites al museu es poden classificar segons diversos criteris relacionats amb el perfil dels visitants i les seves necessitats:
	\begin{itemize}
		\item \texttt{Tipus de grup}:
		\begin{itemize}
			\item Individual
			\item Família
			\item Grup petit (2-5 persones)
			\item Grup gran (més de 5 persones)
		\end{itemize}
		\item \texttt{Durada de la visita}:
		\begin{itemize}
			\item 1 dia
			\item 2 dies
			\item 3 o més dies
		\end{itemize}
		\item \texttt{Coneixement artístic}:
		\begin{itemize}
			\item Baix: coneix poques obres o autors.
			\item Mitjà: coneix algunes èpoques o estils.
			\item Alt: coneix moltes obres i autors, amb preferències clares.
		\end{itemize}
	\end{itemize}
	
	\subsubsection{Sales del museu}
	El museu està organitzat en diferents sales que agrupen les obres segons criteris específics. Algunes de les sales disponibles són:
	\begin{itemize}
		\item \texttt{Sala 1: } 
		\begin{itemize}
			\item Obres destacades:
			\item Autors:
		\end{itemize}
	\end{itemize}
	
	\subsubsection{Quadres}
	
\end{document}